\chapter{Job Scheduler Module User's Manual}
\label{userman}
This chapter explains how to run and configure our scheduler and its other components related to it. 
\section{System Requirements}
\begin{itemize}
\item Intel processor / AMD processor 2.0 GHz or better, RAM: 4 GB, with 500 MB of free storage space.
\item Linux OS (Ubuntu 10.10 / Fedora 13 / Linux mint or better )
\item C++ library
\item SSH services on resources/computers with key based login.
\item MPICH2 library with hydra(mpiexec) \cite{mpich2}
\item GNUPLOT software
\end{itemize}

\section{Installation}
Before executing the scheduler following steps are needed to be done :
\begin{itemize}
\item Extract ``modjs.zip'' 
\subitem \$ tar -xvzf modjs.zip
\item Compile the source code
\subitem \$ make
\item Go to folder ``resource''
\subitem \$ cd resource
\subitem \$ gcc -c resource\_manager.c -o resource
\item Enter available resource information in ``initial\_resource.in'' in specified format.
\item Copy any job file from "data" folder and name it "job.in" for e.g.
\subitem \$ cp data/DAS-2/job\_no\_pred.in job.in
\item prepare hostfile.txt with ip address or domain name of all resources.
\end{itemize}

\section{Execution}
Command to run the job scheduler follows :
\begin{itemize}
 \item mpiexec  -disable-hostname-propagation -hostfile hostfile.txt ./modjsg $<$plot$>$ $<$NUM\_JOBS$>$ $<$POPULATION$>$ $<$GENERATION$>$ $<$RANDOM\_INTEGER$>$
\item $<$plot$>$  - 0 (run without GNUPLOT), 1 (without GNUPLOT)
\item $<$NUM\_JOBS$>$ - Number of jobs to be scheduled on each run of the job scheduler module. Range [100,1000] multiplier of 4.
\item $<$POPULATION$>$ - Chromosome population for genetic algorithm range, Range [200,1000]
 \item $<$GENERATION$>$ - Iteration in genetic algorithm, Range [50,300]
\item $<$RANDOM\_INTEGER$>$ - Any integer for randomize function seed.\\
e.g. \$ mpiexec  -disable-hostname-propagation -hostfile hostfile.txt ./modjsg 0 100 400 300 432421 \\
\item Command to run resource manager : 
\subitem \$ cd resource 
\subitem \$ ./resource 
\end{itemize}

To configure the selection criteria of schedule after final population of efficient schedule has been generated modify weight\_fun[] in report\_best() function in report.c file.
