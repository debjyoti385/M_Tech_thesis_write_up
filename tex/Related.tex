\chapter{Related Work}
Job shop scheduling problem is at least 70 years old. Considerable effort to solve it and find computational complexity has been found to be in 1960 \cite{manne1960job}. This has been proven to be an NP-complete problem in 1979 \cite{jis1979computers}. Many researcher have used heuristic based solving approach to address the problem. Local Search \cite{ritchie2003fast}, Tabu search \cite{abraham2000nature}, simulated Annealing \cite{yarkhan2002experiments} \cite{abraham2000nature} are single heuristic based approach. \\
In Tabu search, one solution $s$ moves to another solution $s'$ located in the neighborhood with a slight modification possible from $s$. Tabu search overcomes the local optimality with a steepest descent/mildest ascent approach. However performance of TS largely depends on the parameters and heuristic used in formulating the problem. For multi-objective scheduling TS might not be sufficient. 

In simulated annealing technique, each solution is mutated and if the mutant spawned exceeds threshold it is rejected, and if less than or equal to the energy of the parent, the difference of threshold and energy of mutant is added to Energy Bank(EB). The  threshold is changed when EB reaches a certain value and population moved to new generation. Simulated annealing mutation/reheating is directly proportional to the energy accumulated in EB. Simulated annealing in Multiobjective domain e.g. AMOSA \cite{bandyopadhyay2008simulated} requires many parameters and domination factor to find near optimal solutions, which are hard to established in grid scheduling.

There are also some hybrid approaches like Tabu search with Ant colony Optimization \cite{ritchie2003static} \cite{ritchie2004hybrid}, GA's with Simulated annealing \cite{zheng2006task}.Other predictive model approaches for the problem are Particle Swarm optimization \cite{liu2010scheduling}  \cite{abraham2006scheduling}, Fuzzy based scheduling \cite{kumar2004fuzzy}. AI based scheduling algorithms like Max-min (Task with more computation time has higher priority), Min-min (Task with less computation time has higher priority), Suffrage (Task with higher sufferage value is given higher priority, its value is determined as the difference of computational time between best and second best resources on which job can be allocated)\cite{ibarra1977heuristic}. All the above work have focused on single objective i.e. minimizing the makespan, which in turn maximizes the utilization of resources. Resource constraint was also not taken into consideration.

Job grouping based scheduling algorithm is used for fine-grained jobs \& light-weight jobs which increase the resource utilization \cite{muthuvelu2005dynamic}  \cite{ang2009bandwidth} . The later have considered communication and bandwidth capabilities. However they have not taken care of predecessor job completion constraint and dynamic behavior of resources in grid.

Genetic algorithms are a stochastic search method introduced in the 1970s in the United States by John Holland [Holland 76] and in Germany by Ingo Rechenberg [Rechenberg 73]. It is based on Darwin's natural selection principle of evolution of biological species. GA operate on a population of solution and apply heuristics such as selection, crossover, and mutation to find better solutions \cite{wall1996genetic}. 

EDSA is a GA’s searching technique in which the crossover and mutation rates are changed dynamically depending on the variances of the fitness values in each generation \cite{yu2008evolution} . The scheduling consider minimization of makespan. 
MOEA  has addressed the need for multi-objective minimization on computational grid, their work was limited to one type of resource, two objectives i.e. makespan \& flowtime, and lacks predecessor job constraint \cite{grosan2007multiobjective}.

In our work based on multi-objective evolutionary algorithm we have converted resource scheduling problem in grid into \emph{resource-constrained project scheduling problem}. We have incorporated dynamic scheduling mechanism, advanced crossover and mutation operator \& minimizing five objectives with pareto front technique. The GA structure of Non-dominating Sorting Genetic Algorithm II proposed by K.Deb \emph{et al.} have helped us in creating the MOJS module \cite{deb2002fast}.

GA based scheduler can act as a real time scheduler due to increase in computational capability of processors in last five years.  We have proposed a job grouping strategy for fine-grained jobs, so that it can deliver job schedule to dispatcher on time. \\
A comparable work with matching constraints could not be found in literature, only few publications deal with multi-objective scheduling \cite{grosan2007multiobjective} but their platform is different from ours. So in result section we experimented our scheduler and produced result on the performance based on various parameters.\\
